%【紙面設定】
%b4,横書き,2段
\documentclass[paper=b4j,landscape,twocolumn,fleqn]{jlreq}
%横書き用の調整
\special{papersize=\the\paperwidth,\the\paperheight}
%間に線をいれる
\setlength{\columnseprule}{0.4pt}
%ページ番号消す
\pagestyle{empty}
%ページ余白の調整
\usepackage[margin=10truemm]{geometry}
%画像挿入用のパッケージ
\usepackage[dvipdfmx]{graphicx}
%数学用のパッケージ
\usepackage{amsmath}
\usepackage{physics}
%見出しによる行取りの数を減らす
\ModifyHeading{section}{lines=1}

%【本文】
\begin{document}
%1段落目
\section{2次元ベクトル基礎}
$\vec{a}(1,2), \vec{b}(3,4), \vec{c}(5,7), \vec{d}(-1,-2), \vec{e}(-3,-7)$とする。
以下の問いに答えよ。\\
 ※$\hat{a}は\vec{a}を正規化したもの(単位ベクトル)とする。$
\begin{align*}
\vec{ab} &= & \vec{ba} &= & \vec{a}+\vec{b} &= & \vec{a}-\vec{b} &= \\
\vec{bc} &= & \vec{cb} &= & \vec{b}+\vec{c} &= & \vec{b}-\vec{c} &= \\ 
\vec{cd} &= & \vec{dc} &= & \vec{c}+\vec{d} &= & \vec{c}-\vec{d} &= \\ 
\vec{de} &= & \vec{ed} &= & \vec{d}+\vec{e} &= & \vec{d}-\vec{e} &= \\
|\vec{a}| &= & |\vec{b}| &= & |\vec{c}| &= & |\vec{d}| &= \\
|\vec{ab}| &= & |\vec{bc}| &= & |\vec{cd}| &= & |\vec{de}| &= \\
\hat{a} &= & \hat{b} &= & \hat{c} &= & \hat{d} &= \\
\hat{ab} &= & \hat{bc} &= & \hat{cd} &= & \hat{de} &= \\
\end{align*}
\section{内積となす角}
$\vec{a}(1,2), \vec{b}(3,4), \vec{c}(5,7), \vec{d}(-1,-2), \vec{e}(-3,-7)$とする。
以下の問いに答えよ。\\
\begin{align*}
\vec{a} \cdot \vec{b} &= & |\vec{a}||\vec{b}| &= & \cos\theta &= &\\
\vec{b} \cdot \vec{c} &= & |\vec{b}||\vec{c}| &= & \cos\theta &= &\\
\vec{c} \cdot \vec{d} &= & |\vec{c}||\vec{d}| &= & \cos\theta &= &\\
\vec{d} \cdot \vec{e} &= & |\vec{d}||\vec{e}| &= & \cos\theta &= &\\
\end{align*}

%2段落目
\newpage
\section{三角比1}
\begin{align*}
  \cos{30} &= & \sin{30} &=\\
  \cos{45} &= & \sin{45} &=\\
  \cos{60} &= & \sin{60} &=\\
  \sin{\frac{1}{6}\pi} &= & \sin{\frac{1}{4}\pi} &=\\
  \sin{\frac{1}{3}\pi} &= & \sin{\frac{1}{2}\pi} &=\\
  \sin{\frac{5}{4}\pi} &= & \sin{\frac{7}{6}\pi} &=\\
  \sin{\pi} &= & \sin{2\pi} &=\\
  \cos{\frac{1}{6}\pi} &= & \cos{\frac{1}{4}\pi} &=\\
  \cos{\frac{1}{3}\pi} &= & \cos{\frac{1}{2}\pi} &=\\
  \cos{\frac{5}{4}\pi} &= & \cos{\frac{5}{6}\pi} &=\\
  \cos{\pi} &= & \cos{2\pi} &=\\
\end{align*}
\section{三角比2}
斜辺c,高さb,底a,底と斜辺のなす角Θ の直角三角形について以下の不明点を答えよ。
\begin{align*}
  a &= & b &= & c &=8 & Θ=30 \\
  a &= & b &= & c &=3 & Θ=45 \\
  a &= & b &= & c &=3 & Θ=60 \\
  a &= & b &= & c &=5 & Θ=30 \\
  a &= & b &= & c &=9 & Θ=45 \\
  a &= & b &= & c &=2 & Θ=60 \\
  a &=3 & b &= & c &=5 & \sin{Θ}= \\
  a &= & b &=3 & c &=7 & \cos{Θ}= \\
  a &=1 & b &=2 & c &= & \sin{Θ}= \\
  a &=3 & b &=4 & c &= & \cos{Θ}= \\
  a &=6 & b &= & c &=12 & \sin{Θ}= \\
  a &= & b &=2 & c &=7 & \cos{Θ}= \\
  a &=4 & b &=5 & c &= & \sin{Θ}= \\
  a &=7 & b &=7 & c &= & \cos{Θ}= \\
\end{align*}
\end{document}


