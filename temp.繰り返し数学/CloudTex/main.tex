%【紙面設定】
%b4,横書き,2段
\documentclass[paper=b4j,landscape,twocolumn,fleqn]{jlreq}
%横書き用の調整
\special{papersize=\the\paperwidth,\the\paperheight}
%間に線をいれる
\setlength{\columnseprule}{0.4pt}
%ページ番号消す
\pagestyle{empty}
%ページ余白の調整
\usepackage[margin=10truemm]{geometry}
%画像挿入用のパッケージ
\usepackage[dvipdfmx]{graphicx}
%数学用のパッケージ
\usepackage{amsmath}
\usepackage{physics}
%見出しによる行取りの数を減らす
\ModifyHeading{section}{lines=1}

%【本文】
\begin{document}
%1段落目
\section{2次元ベクトル基礎}
$\vec{a}(1,2), \vec{b}(3,4), \vec{c}(5,7), \vec{d}(-1,-2), \vec{e}(-3,-7)$とする。
以下の問いに答えよ。\\
 ※$\hat{a}は\vec{a}を正規化したもの(単位ベクトル)とする。$
\begin{align*}
\vec{ab} &= & \vec{ba} &= & \vec{a}+\vec{b} &= & \vec{a}-\vec{b} &= \\
\vec{bc} &= & \vec{cb} &= & \vec{b}+\vec{c} &= & \vec{b}-\vec{c} &= \\ 
\vec{cd} &= & \vec{dc} &= & \vec{c}+\vec{d} &= & \vec{c}-\vec{d} &= \\ 
\vec{de} &= & \vec{ed} &= & \vec{d}+\vec{e} &= & \vec{d}-\vec{e} &= \\
|\vec{a}| &= & |\vec{b}| &= & |\vec{c}| &= & |\vec{d}| &= \\
|\vec{ab}| &= & |\vec{bc}| &= & |\vec{cd}| &= & |\vec{de}| &= \\
\hat{a} &= & \hat{b} &= & \hat{c} &= & \hat{d} &= \\
\hat{ab} &= & \hat{bc} &= & \hat{cd} &= & \hat{de} &= \\
\end{align*}
\section{内積となす角}  
$\vec{a}(1,2), \vec{b}(3,4), \vec{c}(5,7), \vec{d}(-1,-2), \vec{e}(-3,-7)$とする。
以下の問いに答えよ。\\
\begin{align*}
\vec{a} \cdot \vec{b} &= & |\vec{a}||\vec{b}| &= & \cos\theta &= &\\
\vec{b} \cdot \vec{c} &= & |\vec{b}||\vec{c}| &= & \cos\theta &= &\\
\vec{c} \cdot \vec{d} &= & |\vec{c}||\vec{d}| &= & \cos\theta &= &\\
\vec{d} \cdot \vec{e} &= & |\vec{d}||\vec{e}| &= & \cos\theta &= &\\
\end{align*}
\section{垂直条件と射影ベクトル}  
$\vec{a}(1,2), \vec{b}(3,4), \vec{c}(5,7), \vec{d}(-1,-2), \vec{e}(-3,-7)$とする。
以下の問いに答えよ。\\
\begin{align*}
&1. \vec{a} と垂直になるベクトル\\
&2. \vec{b} と垂直になるベクトル\\
&3. \vec{c} と垂直になるベクトル\\
&4. \vec{d} と垂直になるベクトル\\
&5. \vec{e} と垂直になるベクトル\\
&6. \vec{a} から\vec{b}への射影ベクトル\\
&7. \vec{b} から\vec{c}への射影ベクトル\\
&8. \vec{c} から\vec{d}への射影ベクトル\\
&9. \vec{d} から\vec{e}への射影ベクトル\\
&10. \vec{e} から\vec{a}への射影ベクトル\\
\end{align*}

%2段落目
\newpage
\section{速度と加速度}
初速度$V_0(1,2)$,初期位置を$P_0(3,4)$とする。重力は存在しないとする。
\begin{align*}
&1. 2秒後の位置Pを求めよ。\\
&2. 3秒後の位置Pを求めよ。\\
&3. 4秒後の位置Pを求めよ。\\
&4. 加速度A(1,2)とする。2秒後の速度Vを求めよ。\\
&5. 加速度A(2,3)とする。3秒後の速度Vを求めよ。\\
&6. 加速度A(3,4)とする。4秒後の速度Vを求めよ。\\
&7. 加速度A(1,2)とする。2秒後の位置Pを求めよ。\\
&8. 加速度A(2,3)とする。3秒後の位置Pを求めよ。\\
&9. 加速度A(3,4)とする。4秒後の位置Pを求めよ。\\
\end{align*}
\section{放物線運動}
初速度$V_0(1,2)$,初期位置を$P_0(3,4)$,重力加速度を$G(0,-9)$とする。
\begin{align*}
&1. 2秒後の位置Pを求めよ。\\
&2. 3秒後の位置Pを求めよ。\\
&3. 4秒後の位置Pを求めよ。\\
&4. 加速度A(1,2)とする。2秒後の速度Vを求めよ。\\
&5. 加速度A(2,3)とする。3秒後の速度Vを求めよ。\\
&6. 加速度A(3,4)とする。4秒後の速度Vを求めよ。\\
&7. 加速度A(1,2)とする。2秒後の位置Pを求めよ。\\
&8. 加速度A(2,3)とする。3秒後の位置Pを求めよ。\\
&9. 加速度A(3,4)とする。4秒後の位置Pを求めよ。\\
&10. \theta=60度の方向に速さ8で打ち出した。V(x,y)の形に直せ。\\
&11. \theta=60度の方向に速さ8で打ち出した。何秒で頂点に着くか?\\
&12. \theta=60度の方向に速さ8で打ち出した。何秒で地面に着くか?\\
\end{align*}
\section{球の当たり判定}
球Aの位置(1,2),球Bの位置(12,16)とする。半径はそれぞれ1,2とする。
\begin{align*}
&1. Aだけが動くとする。2秒後に当たる為の速度V_aを求めよ。\\
&2. Bだけが動くとする。3秒後に当たる為の速度V_bを求めよ。\\
&3. AもBも動き、Bの速度はAの速度の2倍とする。2秒後に当たる際のそれぞれの速度を求めよ。\\
\end{align*}
\end{document}


