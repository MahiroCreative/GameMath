%【紙面設定】
%b4,横書き,2段
\documentclass[paper=b4j,landscape,twocolumn,fleqn]{jlreq}
%横書き用の調整
\special{papersize=\the\paperwidth,\the\paperheight}
%間に線をいれる
\setlength{\columnseprule}{0.4pt}
%ページ番号消す
\pagestyle{empty}
%ページ余白の調整
\usepackage[margin=10truemm]{geometry}
%画像挿入用のパッケージ
\usepackage[dvipdfmx]{graphicx}
%数学用のパッケージ
\usepackage{amsmath}
%見出しによる行取りの数を減らす
\ModifyHeading{section}{lines=1}

%【本文】
\begin{document}
%1段落目
\section{三平方}
直角三角形の斜辺をc,その他の辺をa,bとする。不明な長さを求めよ。
\begin{align*}
  a&=1 & b&=3 & c&= \\
  a&=2 & b&=6 & c&= \\
  a&=3 & b&=11 & c&= \\
  a&= & b&=3 & c&=8 \\
  a&=1 & b&= & c&=7 \\
  a&=3 & b&= & c&=5 \\
  a&= & b&=4 & c&=11 \\
  a&= & b&=2 & c&=10 \\
  a&=4 & b&= & c&=9 \\
  a&=5 & b&=6 & c&= \\
  a&= & b&=1 & c&=3 \\
  a&=3 & b&=3 & c&= \\
\end{align*}
\section{点と座標}
点a(1,2),点b(2,3),点c(4,6),点d(-2,5)とする。
以下の問いに答えよ。\\
\begin{flushleft}
 1. a,bの中点\\
 2. b,cの中点\\
 3. c,dの中点\\
 4. a,dの中点\\
 5. a,b,cの中点\\
 6. b,c,dの中点\\
 6. a,b,c,dの中点\\
 7. a,bの距離\\
 8. b,cの距離\\
 9. c,dの距離\\
 10. a,dの距離\\
 11. a,cの距離\\
 12. b,dの距離\\
 13. a,bの中点とc,dの中点の距離\\
 14. b,cの中点とa,dの中点の距離\\
 15. b,dの中点とa,cの中点の距離\\
\end{flushleft}

%2段落目
\newpage
\section{三角比1}
\begin{align*}
  \cos{30} &= & \sin{30} &=\\
  \cos{45} &= & \sin{45} &=\\
  \cos{60} &= & \sin{60} &=\\
  \sin{\frac{1}{6}\pi} &= & \sin{\frac{1}{4}\pi} &=\\
  \sin{\frac{1}{3}\pi} &= & \sin{\frac{1}{2}\pi} &=\\
  \sin{\frac{5}{4}\pi} &= & \sin{\frac{7}{6}\pi} &=\\
  \sin{\pi} &= & \sin{2\pi} &=\\
  \cos{\frac{1}{6}\pi} &= & \cos{\frac{1}{4}\pi} &=\\
  \cos{\frac{1}{3}\pi} &= & \cos{\frac{1}{2}\pi} &=\\
  \cos{\frac{5}{4}\pi} &= & \cos{\frac{5}{6}\pi} &=\\
  \cos{\pi} &= & \cos{2\pi} &=\\
\end{align*}
\section{三角比2}
斜辺c,高さb,底a,底と斜辺のなす角Θ の直角三角形について以下の不明点を答えよ。
\begin{align*}
  a &= & b &= & c &=8 & Θ=30 \\
  a &= & b &= & c &=3 & Θ=45 \\
  a &= & b &= & c &=3 & Θ=60 \\
  a &= & b &= & c &=5 & Θ=30 \\
  a &= & b &= & c &=9 & Θ=45 \\
  a &= & b &= & c &=2 & Θ=60 \\
  a &=3 & b &= & c &=5 & \sin{Θ}= \\
  a &= & b &=3 & c &=7 & \cos{Θ}= \\
  a &=1 & b &=2 & c &= & \sin{Θ}= \\
  a &=3 & b &=4 & c &= & \cos{Θ}= \\
  a &=6 & b &= & c &=12 & \sin{Θ}= \\
  a &= & b &=2 & c &=7 & \cos{Θ}= \\
  a &=4 & b &=5 & c &= & \sin{Θ}= \\
  a &=7 & b &=7 & c &= & \cos{Θ}= \\
\end{align*}
\end{document}


