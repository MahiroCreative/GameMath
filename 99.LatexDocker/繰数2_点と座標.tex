%【紙面設定】
%b4,横書き,2段
\documentclass[paper=b4j,landscape,twocolumn,fleqn]{jlreq}
%横書き用の調整
\special{papersize=\the\paperwidth,\the\paperheight}
%間に線をいれる
\setlength{\columnseprule}{0.4pt}
%ページ番号消す
\pagestyle{empty}
%ページ余白の調整
\usepackage[margin=10truemm]{geometry}
%画像挿入用のパッケージ
\usepackage[dvipdfmx]{graphicx}
%数学用のパッケージ
\usepackage{amsmath}
%見出しによる行取りの数を減らす
\ModifyHeading{section}{lines=1}

%【本文】
\begin{document}
%1段落目
\section{中点}
点A(2,3),点B(-2,1),点C(4,-9),点D(-4,-1)とする。
\begin{align*}
 &1. 点Aと点Bの中点\\[1em]
 &2. 点Aと点Cの中点\\[1em]
 &3. 点Aと点Dの中点\\[1em]
 &4. 点Bと点Cの中点\\[1em]
 &5. 点Cと点Dの中点\\[1em]
 &6. 点A,B,Cの中点\\[1em]
 &7. 点B,C,Dの中点\\[1em]
 &8. 点A,B,C,Dの中点\\[1em]
\end{align*}
\section{2点間の距離}
点A(2,3),点B(-2,1),点C(4,-9),点D(-4,-1)とする。
\begin{align*}
 &1. 点Aから点Bの距離\\[1em]
 &2. 点Bから点Cの距離\\[1em]
 &3. 点Cから点Dの距離\\[1em]
 &4. 点A,Bの中点と点C,Dの中点の距離\\[1em]
 &5. 点B,Cの中点と点D,Aの中点の距離\\[1em]
 &6. 点A,B,Cの中点と点Dの距離\\[1em]
 &7. 点B,C,Dの中点と点Aの距離\\[1em]
 &8. 点A,B,C,Dの中点と原点の距離\\[1em]
\end{align*}

%2段落目
\newpage
\section{2点間の移動}
点A(2,3),点B(-2,1),点C(4,-9),点D(-4,-1)とする。
\begin{align*}
 &1. 点Aから点Bに2秒かけて移動する。1秒ごとのx,yそれぞれの移動量を求めよ。\\[1em]
 &2. 点Aから点Cに4秒かけて移動する。1秒ごとのx,yそれぞれの移動量を求めよ。\\[1em]
 &3. 点Aから点Dに2秒かけて移動する。1秒ごとのx,yそれぞれの移動量を求めよ。\\[1em]
 &4. 点Bから点Cに4秒かけて移動する。1秒ごとのx,yそれぞれの移動量を求めよ。\\[1em]
 &5. 点Bから点Dに2秒かけて移動する。1秒ごとのx,yそれぞれの移動量を求めよ。\\[1em]
 &6. 点Cから点Dに4秒かけて移動する。1秒ごとのx,yそれぞれの移動量を求めよ。\\[1em]
 &7. 点Bから点Aに2秒かけて移動する。1秒ごとのx,yそれぞれの移動量を求めよ。\\[1em]
 &8. 点Cから点Aに4秒かけて移動する。1秒ごとのx,yそれぞれの移動量を求めよ。\\[1em]
\end{align*}
\section{円の当たり判定}
円A:中点(1,2),半径1。円B:中点(4,8),半径2。円C:中点(0,1),半径3。とする。
\begin{align*}
 &1. 円Aと円Bは当たっているか? 当たっていた場合、離れるための移動量を求めよ。\\[1em]
 &2. 円Aと円Cは当たっているか? 当たっていた場合、離れるための移動量を求めよ。\\[1em]
 &3. 円Aが動き、円Bに2秒後に接触するとする。1秒ごとのx,yそれぞれの移動量を求めよ。\\[1em]
 &4. 円Bが動き、円Cに4秒後に接触するとする。1秒ごとのx,yそれぞれの移動量を求めよ。\\[1em]
\end{align*}
\section{円の当たり判定の一般化}
円A:中点(a,b),半径r。円B:中点(c,d),半径R。とする。
\begin{align*}
 &1. 根号を用いて円Aと円Bの当たり判定の不等式を立てよ。\\[1em]
 &2. 根号を用いず円Aと円Bの当たり判定の不等式を立てよ。\\[1em]
\end{align*}
\end{document}


