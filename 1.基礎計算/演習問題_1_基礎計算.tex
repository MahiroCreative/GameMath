%【紙面設定】
%b4,横書き,2段
\documentclass[paper=b4j,landscape,twocolumn,fleqn]{jlreq}
%横書き用の調整
\special{papersize=\the\paperwidth,\the\paperheight}
%間に線をいれる
\setlength{\columnseprule}{0.4pt}
%ページ番号消す
\pagestyle{empty}
%ページ余白の調整
\usepackage[margin=10truemm]{geometry}
%画像挿入用のパッケージ
\usepackage[dvipdfmx]{graphicx}
%数学用のパッケージ
\usepackage{amsmath}
%見出しによる行取りの数を減らす
\ModifyHeading{section}{lines=1}

%【本文】
\begin{document}
%1段落目
\section{五則演算}
\begin{align*}
  4+8&= & 12+88&= & 128+256&=\\
  4-8&= & 32-12&= & 512-256&=\\
  2*8&= & 16*24&= & 128*256&=\\
  8/2&= & 24*8&= & 512*128&=\\
  7\%2&= & 25\%11&= & 512\%100&=\\
\end{align*}
\section{累乗計算}
\begin{align*}
  2^0&= & 2^2&= & 2^3&=\\
  3^0&= & 3^2&= & 3^3&=\\
  4^0&= & 4^2&= & 4^3&=\\
  5^0&= & 5^2&= & 5^3&=\\
  6^0&= & 6^2&= & 6^3&=\\
  7^0&= & 7^2&= & 7^3&=\\
  8^0&= & 8^2&= & 8^3&=\\
  9^0&= & 9^2&= & 9^3&=\\
\end{align*}
\section{平方根}
\begin{align*}
  \sqrt{4}&= & \sqrt{8}&= & \sqrt{18}&=\\
  \sqrt{9}&= & \sqrt{12}&= & \sqrt{27}&=\\
  \sqrt{16}&= & \sqrt{32}&= & \sqrt{64}&=\\
  \sqrt{25}&= & \sqrt{20}&= & \sqrt{45}&=\\
  \sqrt{36}&= & \sqrt{24}&= & \sqrt{54}&=\\
  \sqrt{49}&= & \sqrt{28}&= & \sqrt{63}&=\\
  \sqrt{64}&= & \sqrt{128}&= & \sqrt{256}&=\\
  \sqrt{81}&= & \sqrt{36}&= & \sqrt{72}&=\\
\end{align*}

%2段落目
\newpage
\section{指数計算1}
\begin{align*}
  2^{-1}&= & 2^{-2}&= & 2^{\frac{1}{2}}&=\\
  3^{-1}&= & 3^{-2}&= & 3^{\frac{1}{2}}&=\\
  4^{-1}&= & 4^{-2}&= & 4^{\frac{1}{2}}&=\\
  5^{-1}&= & 5^{-2}&= & 5^{\frac{1}{2}}&=\\
  6^{-1}&= & 6^{-2}&= & 6^{\frac{1}{2}}&=\\
  7^{-1}&= & 7^{-2}&= & 7^{\frac{1}{2}}&=\\
  8^{-1}&= & 8^{-2}&= & 8^{\frac{1}{2}}&=\\
  9^{-1}&= & 9^{-2}&= & 9^{\frac{1}{2}}&=\\
\end{align*}
\section{指数計算2}
\begin{align*}
  2^2*2^3&= & 2^2*2^{-3}&= & 2^2*2^{\frac{1}{2}}&=\\
  3^2*3^4&= & 3^2*3^{-4}&= & 3^2*3^{\frac{1}{2}}&=\\
  4^3*2^5&= & 4^3*4^{-5}&= & 4^2*4^{\frac{1}{2}}&=\\
  8^2*8^3&= & 8^2*8^{-3}&= & 8^2*8^{\frac{1}{2}}&=\\
  6^2*6^4&= & 6^2*6^{-4}&= & 6^2*6^{\frac{1}{2}}&=\\
  8^3*2^5&= & 8^3*4^{-5}&= & 8^2*4^{\frac{1}{2}}&=\\
\end{align*}
\section{指数計算3}
\begin{align*}
  2^2*8*\frac{1}{2} &= & 2^2*16*2^{\frac{1}{2}}= \\
  3^2*9*\frac{1}{2} &= & 3^2*81*2^{\frac{1}{2}}= \\
  \sqrt{8}*2^{\frac{1}{2}}*2^{-\frac{1}{2}} &= & \sqrt{16}*4^{\frac{1}{2}}*4^{-\frac{1}{2}}&= \\
  \sqrt{12}*3^{\frac{1}{2}}*4^{-\frac{1}{2}} &= & \sqrt{18}*3^{-\frac{1}{2}}*8^{-\frac{1}{2}}&= \\
  \sqrt{18}*2^{\frac{1}{2}}*8^{-\frac{1}{2}} &= & \sqrt{36}*9^{\frac{1}{2}}*9^{-\frac{1}{2}}&= \\
  \sqrt{81}*9^{\frac{1}{2}}*9^{-\frac{1}{2}} &= & \sqrt{32}*3^{\frac{1}{2}}*5^{-\frac{1}{2}}&= \\
  \sqrt{18}*4^{\frac{1}{2}}*8^{-\frac{1}{2}} &= & \sqrt{36}*6^{-\frac{1}{2}}*6^{-\frac{1}{2}}&= \\
\end{align*}
\end{document}


